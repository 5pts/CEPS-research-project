\documentclass[12pt]{article}
\usepackage[utf8]{inputenc}
\usepackage{amsmath, amssymb}
\usepackage{geometry}
\geometry{a4paper, margin=2.5cm}
\title{模型构建说明(严格版)}
\date{}

\begin{document}
\maketitle

\section*{1. 因变量定义}
教育期望变量来自学生问卷 w2b18,取值 1--10:
\[
Y \in \{1,2,3,4,5,6,7,8,9,10\}
\]
其中 10 表示“无所谓”,不属于教育路径的有序等级。主模型中将 10 视为非有序响应并剔除,保留 1--9 构成有序因变量。

\section*{2. 自变量与标准化}
主要自变量包括:
同辈关系指数(bonding\_idx,横向;w2b0605/0606/0607)、师生关系指数(linking\_idx,纵向;w2b0507/0508/0509 + w2c09,并对相加结果再 z-score)、SES(ses\_pca:父母教育、家庭经济、藏书量、是否有书桌、是否有电脑的 PCA 指数;其中 has\_computer 做 1-值反向处理)、户口类型(hukou\_type)、认知得分(cog\_score,0视为未参加测试)。

对连续变量进行 z 标准化:
\[
z(x_i) = \frac{x_i - \bar{x}}{s_x}
\]
其中 $\bar{x}$ 为样本均值,$s_x$ 为样本标准差。二分变量 hukou\_type 不标准化。

\section*{3. 主模型:有序Logit}
对 $k = 1,\dots,8$,有序Logit模型为:
\[
P(Y_i \le k \mid x_i) = \Lambda(\theta_k - x_i^\top \beta)
\]
其中 $\Lambda(z) = \frac{1}{1+e^{-z}}$,$\theta_k$ 为阈值参数,$\beta$ 为回归系数。

该模型隐含比例优势假设(Proportional Odds, PO),即 $\beta$ 在不同阈值方程中保持一致。

\section*{4. 班级聚类稳健标准误}
考虑同班学生存在相关性,采用班级聚类稳健标准误:
\[
\widehat{V}_{CR} = (X'WX)^{-1}\left(\sum_{g=1}^{G} X_g' u_g u_g' X_g\right)(X'WX)^{-1}
\]
其中 $g$ 表示班级,$u_g$ 为得分残差,$X_g$ 为班级内设计矩阵。

\section*{5. 比例优势假设检验}
使用有序Logit与多项Logit的似然比检验:
\[
LR = 2(\ell_{MNLogit} - \ell_{Ordered})
\]
若 $p<0.05$,说明PO假设可能不成立。此时建议以多项Logit作为稳健性对照,并解释其与有序Logit结论的一致性与差异。

\section*{6. “无所谓(10)”的处理与敏感性}
为评估剔除 10 的影响,对 10 组与 1--9 组做均值差异检验:
\begin{itemize}
  \item 连续变量:Welch t 检验
  \item 二分变量:$\chi^2$ 检验
\end{itemize}
若差异显著,应在正文中注明:主模型解释对象为“具有明确教育路径的学生”。

\section*{7. 探索性随机森林(非因果解释)}
随机森林用于探索非线性与变量重要性,不作为因果结论依据。分类目标为 1--9。

\section*{8. 示例代码}
有序Logit:
\begin{verbatim}
from statsmodels.miscmodels.ordinal_model import OrderedModel
X = df[["bonding_idx_z","linking_idx_z","ses_pca_z",
        "hukou_type","cog_score_z"]]
y = df["expect_edu_raw"]  # 1-9
model = OrderedModel(y, X, distr="logit")
res = model.fit(method="bfgs", disp=False)
\end{verbatim}

聚类稳健标准误:
\begin{verbatim}
from statsmodels.stats.sandwich_covariance import cov_cluster
import numpy as np
cov = cov_cluster(res, df["clsids"])
se = np.sqrt(np.diag(cov))
\end{verbatim}

\section*{9. 输出文件}
主模型与检验结果:results/phase3/ordinal_model_report.txt\\
随机森林重要性:results/phase3/ordinal_rf_feature_importance.csv\\
图表:figures/report_phase3/ 与 figures/ordinal_rf/

\end{document}
