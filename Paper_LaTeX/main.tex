% !TEX program = xelatex
% 使用官方 APA7 文档类 (stu = 学生论文,符合双倍行距标准)
\documentclass[stu, 12pt, floatsintext]{apa7}

% =============================================
% English support and fonts
% =============================================
\usepackage[american]{babel}
\usepackage{csquotes}
\DeclareLanguageMapping{american}{american-apa}
\usepackage{newtxtext,newtxmath} % 使用类 Times New Roman 字体

% =============================================
% 参考文献配置 (APA 7th 标准)
% =============================================
\usepackage[style=apa, sortcites=true, sorting=nyt, backend=biber]{biblatex}
\addbibresource{references.bib} % 加载参考文献文件

% =============================================
% 其他常用宏包
% =============================================
\usepackage{graphicx}  % 图片
\usepackage{booktabs}  % 三线表
\usepackage{float}     % 图片位置

% =============================================
% 论文元数据 (Metadata)
% =============================================
\title{Your Paper Title Here}
% APA 7 student papers do not use a running head
\author{Your Name}
\affiliation{Your Institution}
\course{Course Name}
\professor{Instructor Name}
\duedate{Due Date}

% 摘要
\abstract{
    Write the abstract here. In APA format, the abstract should not be indented.
    
    This study uses CEPS data and a GEE model to test whether school social capital compensates for family disadvantage...
}

\keywords{school social capital, educational expectations, compensatory effect, CEPS}

\begin{document}

% 生成封面
\maketitle

% =============================================
% 正文区域 (自动双倍行距)
% =============================================

\section{Introduction}
In the context of China’s educational stratification, the Hukou (household registration) system remains a persistent institutional barrier, creating a dual structure that disadvantages rural-to-urban migrant children. While compulsory education policies have expanded access, significant disparities persist in educational expectations—a key psychological precursor to attainment—between migrant and local students \parencite{wu2010economic, xie2014income}. Unlike academic achievement, which reflects past performance, educational expectations capture students’ perceived future possibilities and are highly sensitive to their social environment. Understanding the drivers of these expectations is essential, particularly for migrant adolescents who must navigate a system that often limits their access to urban high schools.

To explain variations in educational expectations, scholars widely utilize Social Capital Theory. However, a nuanced understanding requires distinguishing between the types of social ties. Drawing on Stanton-Salazar's \parencite{stanton2011social} framework of institutional agents, this study operationalizes social capital into two distinct forms: bonding social capital and bridging social capital. Here, bonding refers to horizontal ties among peers, which provide emotional solidarity. In contrast, bridging is defined as vertical ties with teachers—institutional agents who possess dominant cultural capital and information necessary to navigate the educational hierarchy. While peer cliques (bonding) may sometimes reinforce anti-school subcultures, teacher-student interactions (bridging) serve as vital conduits for resources that are otherwise inaccessible to disadvantaged families.

Despite the extensive use of the China Education Panel Survey (CEPS) in recent literature, few studies have empirically tested the heterogeneous effects of these two forms of social capital. Most existing research aggregates social capital into a single construct or focuses primarily on family SES \parencite{sun2020family}, failing to disentangle whether peer bonding and teacher bridging function differently for distinct student groups. This gap is significant: for migrant students lacking local roots, does strong bonding with other migrants lead to social segmentation from the mainstream, while bridging with teachers acts as a necessary compensatory mechanism? The interaction between social capital types and institutional status remains under-explored. This study addresses these issues by utilizing national baseline data from the CEPS. We structurally distinguish between bonding (peer) and bridging (teacher) social capital to examine their distinct impacts on educational expectations. Crucially, we incorporate Hukou status as a key moderator, investigating whether the marginal return of bridging social capital is greater for migrant students compared to their local counterparts, while strictly controlling for family SES to isolate the institutional effect. By doing so, this paper seeks to reveal how specific relational resources can either mitigate or exacerbate the structural inequalities imposed by the Hukou system.

\section{Literature Review}
Prior research on social capital is discussed in \parencite{coleman1988social}. Related work is also reviewed by \textcite{liu2015school}.

\section{Data and Methods}
\subsection{Data Source}
This study uses CEPS data...

\subsection{Statistical Model}
We use generalized estimating equations (GEE) for analysis \parencite{liang2017gee}.

\section{Results}
The results show...

% 插入图片示例
\begin{figure}[h]
    \centering
    \includegraphics[width=0.8\textwidth]{figures/interaction_plot.png}
    \caption{Interaction effect of student-teacher relations and family SES}
    \label{fig:interaction}
\end{figure}

\section{Discussion}
This study confirms...

% =============================================
% 参考文献列表 (自动生成)
% =============================================
\printbibliography

\end{document}
