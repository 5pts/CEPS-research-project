% !TEX program = xelatex
\documentclass[12pt,a4paper]{article}
\usepackage[margin=1in]{geometry}
\usepackage{xeCJK}
\setCJKmainfont{SimSun}
\setmainfont{Times New Roman}
\usepackage{amsmath,amssymb}
\usepackage{graphicx}
\usepackage{booktabs}
\usepackage{hyperref}
\usepackage{setspace}
\usepackage{enumitem}
\usepackage{tcolorbox}
\usepackage{xcolor}

\hypersetup{colorlinks=true,linkcolor=blue,urlcolor=blue}
\onehalfspacing

\title{汇报讲解稿\\(逐页对应 report\_summary.pdf,19 页)}
\author{面向口播/演示的逐页解说}
\date{\today}

\begin{document}

\maketitle
\tableofcontents
\newpage

%==============================================================================
\section{第 1 页:封面、摘要、假设总览}
%==============================================================================
\subsection{页面内容}
\begin{itemize}
  \item 标题、摘要;研究目标与假设列表 H1/H2a–c/H3。
\end{itemize}
\subsection{讲解话术}
\begin{tcolorbox}[colback=blue!5,colframe=blue!60!black,title=开场与摘要]
“本报告用 CEPS 八年级数据(有效样本≈9,000,剔除‘10=无所谓’),看学校社会资本如何影响教育期望。主效应:师生联结(Linking)和同伴纽带(Bonding)都显著正向。事前假设是‘补偿效应’,但 Bonding×SES 反而指向优势累积(高 SES 获益更多,p≈0.052);Linking×SES 不显著;Linking×Rural 弱正向未显著。PO 假设被拒绝;Linking 在低段有非线性(样条 p≈0.005)。H3 探索:比较 Linking vs Bonding 的补偿力度,是否对弱势更有帮助或至少不放大差距。”
\end{tcolorbox}

%==============================================================================
\section{第 2 页:数据来源与样本清洗}
%==============================================================================
\subsection{页面内容}
\begin{itemize}
  \item 数据:CEPS Wave 2,PPS 抽样,全国代表性。
  \item 样本处理:剔除 “10=无所谓”,N≈9,012;保留班级 ID 做聚类稳健 SE。
\end{itemize}
\subsection{讲解话术}
\begin{tcolorbox}[colback=blue!5,colframe=blue!60!black,title=样本与清洗]
“我们用 CEPS 第二轮八年级追踪。因变量是 1–9 的期望等级,10=‘无所谓’无法排序,直接剔除。最终样本约 9,012;班级号 clsids 用于聚类稳健标准误,防止班内相关性低估 SE。若‘无所谓’人群有系统偏差,后续可用选择模型或多项 Logit 做稳健性。”
\end{tcolorbox}

%==============================================================================
\section{第 3 页:教育期望编码表(Table 1)}
%==============================================================================
\subsection{页面内容}
\begin{itemize}
  \item Table 1:w2b18 选项 1–9(不读/小学 … 博士),10=无所谓已剔除。
  \item 引出核心自变量 Bonding/Linking。
\end{itemize}
\subsection{讲解话术}
\begin{tcolorbox}[colback=green!5,colframe=green!50!black,title=Table 1 讲解]
“问卷问:‘你希望最高读到什么学历?’ 编码 1–9 从不读/小学到博士,10=‘无所谓’仅列示但不入模。这样保持标签严格有序,支持有序 Logit 和序数分类。”
\end{tcolorbox}

%==============================================================================
\section{第 4 页:Bonding \& Linking 问卷条目表}
%==============================================================================
\subsection{页面内容}
\begin{itemize}
  \item 新增表:w2b0605/06/07,w2b0507/08/09,w2c09 的题干与 1–4 频率选项;反向计分与二元化规则。
\end{itemize}
\subsection{讲解话术}
\begin{tcolorbox}[colback=green!5,colframe=green!50!black,title=Bonding/Linking 题目]
“Bonding:w2b0605 反向(经常有麻烦/被欺负),0606 友善,0607 归属感,频率 1–4。Linking:w2b0507–0509 老师表扬频率(1–4),w2c09 主动与老师谈话(1=经常,其余归 0)。先反向,再合成,最终 Z-score。”
\end{tcolorbox}

%==============================================================================
\section{第 5 页:PCA 方法与碎石图(Figure 1)}
%==============================================================================
\subsection{页面内容}
\begin{itemize}
  \item SES PCA 文本解释;Figure 1 碎石图(PC1 44.1\%,拐点后仅保留 PC1)。
\end{itemize}
\subsection{讲解话术}
\begin{tcolorbox}[colback=yellow!5,colframe=orange!60!black,title=Figure 1 碎石图]
“PCA 把父母学历、家庭经济、藏书、书桌、电脑 5 个相关指标压成综合 SES。碎石图显示 PC1 解释 44.1\% 方差,PC2 后迅速下跌,拐点在 PC1 之后,因此保留 PC1 作为 SES 综合分。”
\end{tcolorbox}

%==============================================================================
\section{第 6 页:PC1 载荷与 SES 分布(Figure 2, 3)}
%==============================================================================
\subsection{页面内容}
\begin{itemize}
  \item Figure 2:PC1 载荷条形图(全为正,藏书/书桌最高)。
  \item Figure 3:SES 得分分布(近似正态,均值 0)。
\end{itemize}
\subsection{讲解话术}
\begin{tcolorbox}[colback=yellow!5,colframe=orange!60!black,title=Figure 2 \& 3]
“PC1 载荷全正,说明是综合 SES 轴;藏书和书桌贡献最大,其次父母学历、经济,电脑最小。SES 得分近似正态,便于下游模型线性/树模型使用。”
\end{tcolorbox}

%==============================================================================
\section{第 7 页:PCA 输入相关矩阵(Figure 4)}
%==============================================================================
\subsection{页面内容}
\begin{itemize}
  \item Figure 4:五个 SES 指标的相关矩阵,呈中度正相关。
\end{itemize}
\subsection{讲解话术}
\begin{tcolorbox}[colback=yellow!5,colframe=orange!60!black,title=Figure 4]
“各指标有中度正相关,验证了用 PCA 降维的必要性,避免多重共线性拖累回归。”
\end{tcolorbox}

%==============================================================================
\section{第 8–9 页:统计模型与因变量分布(Figure 5)}
%==============================================================================
\subsection{页面内容}
\begin{itemize}
  \item Ordered Logit 定义、切点、累积概率公式;交互项写法。
  \item Figure 5:教育期望分布(1–9,左偏,本科最多)。
\end{itemize}
\subsection{讲解话术}
\begin{tcolorbox}[colback=blue!5,colframe=blue!60!black,title=模型与分布]
“Ordered Logit 适合有序标签;切点决定等级边界。交互示例:Bonding×SES,β\textsubscript{3}>0 表示高 SES 获益更多。因变量分布左偏,本科占比最高,硕士、大专次之。”
\end{tcolorbox}

%==============================================================================
\section{第 10–11 页:自变量分布与相关矩阵(Figure 6, 7, 8)}
%==============================================================================
\subsection{页面内容}
\begin{itemize}
  \item Figure 6:Bonding 分布(标准化,多峰源于离散问卷)。
  \item Figure 7:Linking 分布(标准化,多峰)。
  \item Figure 8:核心变量相关矩阵(Bonding–Linking 中度相关;认知–期望最强;户籍最弱)。
\end{itemize}
\subsection{讲解话术}
\begin{tcolorbox}[colback=yellow!5,colframe=orange!60!black,title=Figure 6–8]
“Bonding/Linking 多峰是因为原始题目是 4 级频率。核心变量相关:Bonding 与 Linking r≈0.46,说明有重叠但非同一概念;认知与教育期望 r≈0.30 最强;户籍相关最弱且略负。”
\end{tcolorbox}

%==============================================================================
\section{第 11–12 页:回归结果(表)}
%==============================================================================
\subsection{页面内容}
\begin{itemize}
  \item Ordered Logit 主效应表:Linking 0.19,Bonding 0.14,SES 0.33,Cognitive 0.56,Hukou 0.11;MLE,聚类 SE 方向不变。表下注明主效应模型;含交互模型系数仍约 0.19/0.14。
\end{itemize}
\subsection{讲解话术}
\begin{tcolorbox}[colback=green!5,colframe=green!50!black,title=主效应表]
“系数为 log-odds:Linking/Bonding 均显著正向;认知 0.56 最大,SES 0.33 次之;户籍 0.11 弱正向。表中为主效应模型,含交互模型下 Linking/Bonding 量级相近且仍显著。”
\end{tcolorbox}

%==============================================================================
\section{第 12–13 页:交互效应(Figure 9, 10, 11)}
%==============================================================================
\subsection{页面内容}
\begin{itemize}
  \item 文字:Bonding×SES 正向(MLE p≈0.015,聚类 p≈0.052,趋势);Linking×SES 不显著;Linking×Rural 正但 p≈0.16。
  \item 图:Figure 9 Bonding×SES 分组概率;Figure 10 Linking×SES;Figure 11 Linking×Rural。
\end{itemize}
\subsection{讲解话术}
\begin{tcolorbox}[colback=yellow!5,colframe=orange!60!black,title=交互解读]
“Bonding×SES 显示高 SES 获益更大,违背补偿,接近马太效应;稳健 SE 边界显著。Linking×SES 近似零,说明师生关系对不同 SES 一视同仁。Linking×Rural 正向但不显著,弱补偿迹象有限。”
\end{tcolorbox}

%==============================================================================
\section{第 13 页:非线性与稳健性(Figure 12)}
%==============================================================================
\subsection{页面内容}
\begin{itemize}
  \item PO 检验被拒;未实现 partial PO,附录有多项 Logit 对照。
  \item Figure 12:Linking 样条效应(df=4,degree=3,低段更强,p≈0.005);Bonding 无非线性。
\end{itemize}
\subsection{讲解话术}
\begin{tcolorbox}[colback=yellow!5,colframe=orange!60!black,title=非线性]
“PO 被拒,需提醒阈值异质性。样条检验显示 Linking 在低水平的边际效应更大,说明弱联系时提升最明显;Bonding 没有类似非线性。”
\end{tcolorbox}

%==============================================================================
\section{第 14–18 页:机器学习补充(Figure 13, 14)}
%==============================================================================
\subsection{页面内容}
\begin{itemize}
  \item Figure 13:RF 流程图(数据→标准化/编码→PCA SES→RF;200 树、深度 5、叶子≥80、random\_state=42;未做 train/val 或 CV,评估节点标记 planned)。
  \item 变量重要性条形图(Figure 14):认知 0.409,SES 0.378,Linking 0.133,Bonding 0.070,Hukou 0.010。
  \item 文本:RF 与回归方向一致,但未评估预测性能,属探索性、非因果;提示可补 PDP/Permutation 与 CV。
\end{itemize}
\subsection{讲解话术}
\begin{tcolorbox}[colback=blue!5,colframe=blue!60!black,title=RF 流程与结果]
“随机森林用全量数据做模式识别,不做因果推断。参数定死(200 树,深度 5,叶 80)。重要性排序与回归一致:认知/SES 领先,Linking 高于 Bonding,户籍最弱。由于缺少 train/val/CV 和类权重,性能可能乐观;后续需补评估、PDP、Permutation。”
\end{tcolorbox}

%==============================================================================
\section{第 19 页:结论、局限、后续方向}
%==============================================================================
\subsection{页面内容}
\begin{itemize}
  \item 结论要点:主效应正向;Bonding×SES 优势累积;Linking×SES ns;Linking×Rural 弱;PO 被拒;Linking 低段非线性;RF 对照一致但仅模式参考。
  \item 局限:剔除“无所谓”潜在偏差;未做 partial PO;样条设定依赖;RF 无 CV/类权重;量表维度有限。
  \item 后续方向:partial PO/广义有序;对“无所谓”做选择模型或 MI;扩展非线性/高阶交互;补齐 ML 评估与 PDP/Permutation;采集 Bridging/学校层级社会资本。
\end{itemize}
\subsection{讲解话术}
\begin{tcolorbox}[colback=green!5,colframe=green!50!black,title=收束与展望]
“主效应稳健,交互揭示 Bonding 的优势累积风险,Linking 更均衡。局限与未来工作已点明:补做 partial PO、选择模型、非线性稳健、ML 评估与可视化,以及补充跨层社会资本指标,才能把‘补偿 vs 累积’的机制讲得更透。”
\end{tcolorbox}

\end{document}
