\documentclass[12pt, a4paper]{article}
\usepackage[UTF8, heading = true]{ctex}
\usepackage{geometry}
\usepackage{amsmath, amssymb}
\usepackage{booktabs}
\usepackage{longtable}
\usepackage{hyperref}
\usepackage{graphicx}
\usepackage{float}
\usepackage{enumitem}
\usepackage{color}

% 页面设置
\geometry{left=2.5cm, right=2.5cm, top=2.5cm, bottom=2.5cm}
\linespread{1.5}

% 标题信息
\title{\textbf{学校社会资本与初中生教育期望:\\基于CEPS数据的实证研究报告}}
\author{社会科学方法论课程组}
\date{\today}

\begin{document}

\maketitle

\begin{abstract}
本研究利用中国教育追踪调查(CEPS)初中基线数据($N \approx 9314$),旨在探究学校场域内的两种“人际关系资本”——同伴关系(Bonding Social Capital)与师生关系(Linking Social Capital)对初中生教育期望的影响。研究采用有序Logit模型(Ordered Logit Model)及聚类稳健标准误(Cluster-Robust SE),实证结果表明:(1) 在控制认知能力与家庭背景后,同伴关系与师生关系均能显著提升学生的教育期望;(2) 交互效应分析揭示了同伴关系的“马太效应”,即高家庭社会经济地位(SES)学生从中获益更多,而师生关系对农村学生呈现出微弱的“补偿效应”;(3) 探索性机器学习分析进一步验证了各因素的重要性排序。本报告详细阐述了从变量构造、模型选择到结果解释的全过程,并讨论了研究的局限性。

\textbf{关键词:} 教育期望;社会资本;有序Logit;马太效应;CEPS
\end{abstract}

\newpage
\tableofcontents
\newpage

\section{研究背景与问题提出}

初中阶段是青少年教育分流的关键时期,个体的\textbf{教育期望(Educational Expectation)}——即学生对自己未来能达到最高学历的预期——对其学业成就具有重要的导向作用。除了众所周知的家庭社会经济地位(SES)和个人认知能力外,学校环境中的人际关系是否构成了另一种重要的资源?

本研究引入社会资本理论,将学校人际关系划分为两个维度:
\begin{enumerate}
    \item \textbf{同伴关系(Bonding Social Capital / 横向关系)}:指同学间的紧密程度、友好互动及班级归属感。
    \item \textbf{师生关系(Linking Social Capital / 纵向关系)}:指掌握制度性资源的教师给予学生的情感支持、表扬与关注。
\end{enumerate}

\textbf{核心研究问题:}
\begin{itemize}
    \item \textbf{主效应 ($H_1$)}:这两种关系能否独立提升初中生的教育期望?
    \item \textbf{交互效应 ($H_2$)}:这种提升作用在不同阶层(高低SES)或户籍(城乡)学生中是否公平?是存在“马太效应”(优势累积),还是“补偿效应”(雪中送炭)?
\end{itemize}

\section{数据来源与样本筛选}

\subsection{数据来源}
数据来源于\textbf{中国教育追踪调查(CEPS)}初中基线调查数据。该数据具有全国代表性,包含了多阶段概率抽样抽取的初中生样本。

\subsection{样本清洗流程}
\begin{enumerate}
    \item \textbf{初始样本}:包含约 9827 名学生。
    \item \textbf{关键剔除}:剔除在教育期望问题(w2b18)中回答“10=无所谓”的学生($N=434$)。
    \begin{itemize}
        \item \textbf{理由}:有序Logit模型要求因变量具有严格的等级顺序($1<2<\dots<9$)。“无所谓”代表退出教育竞争态度,而非更高等级。
        \item \textbf{选择性偏差说明}:经 $t$ 检验和 $\chi^2$ 检验,被剔除组在 SES、认知得分、同伴及师生关系得分上均显著低于保留组。因此,本研究结论仅适用于\textbf{有明确升学意愿}的学生群体。
    \end{itemize}
    \item \textbf{最终样本}:进入主模型分析的有效样本量约为 \textbf{9314} 人。
\end{enumerate}

\section{变量测量与构造}

\subsection{因变量:教育期望 ($Y$)}
基于问卷题目 \texttt{w2b18},将选项编码为 1-9 的定序变量:
\begin{quote}
1=不念了, 2=初中, 3=中专/技校, 4=职高, 5=普高, 6=大专, 7=本科, 8=研究生, 9=博士。
\end{quote}
描述性统计显示,数据呈左偏分布,选择“本科”(7)的人数最多($N=3429$),其次为“研究生”(8)和“博士”(9)。

\subsection{自变量:学校社会资本}
所有连续变量在合成后均进行 Z-score 标准化处理。

\subsubsection{同伴关系指数 ($Bonding\_Idx$)}
\begin{itemize}
    \item \textbf{指标来源}:\texttt{w2b0605}(同学友好)、\texttt{w2b0606}(班风好)、\texttt{w2b0607}(常参加活动)。
    \item \textbf{构造方法}:取三者均值/加总后标准化。
\end{itemize}

\subsubsection{师生关系指数 ($Linking\_Idx$)}
\begin{itemize}
    \item \textbf{指标来源}:
    \begin{itemize}
        \item 表扬维度 ($Praise$):\texttt{w2b0507}, \texttt{w2b0508}, \texttt{w2b0509} 的均值。
        \item 谈心维度 ($Talk$):\texttt{w2c09}(1=是, 0=否)。
    \end{itemize}
    \item \textbf{构造方法}:$Linking = Z(Z(Praise) + Z(Talk))$。
\end{itemize}

\subsection{控制变量}
\begin{itemize}
    \item \textbf{家庭背景 ($SES\_PCA$)}:选取父母学历、家庭经济自评、藏书量、书桌、电脑拥有情况,通过主成分分析(PCA)提取第一主成分并标准化。
    \item \textbf{户籍 ($Hukou$)}:二分变量,0=城镇,1=农村。
    \item \textbf{认知能力 ($Cog\_Score$)}:标准化后的认知测试得分(0分视为缺失)。
\end{itemize}

\section{统计模型与方法}

由于因变量 $Y$ 为定序变量,本研究采用 \textbf{有序Logit模型(Ordered Logit Model)}。

\subsection{模型设定}
设 $\theta_k$ 为第 $k$ 个等级的切点($k=1,\dots,8$),模型公式为:
\begin{equation}
    \ln \left( \frac{P(Y \le k | \mathbf{x})}{P(Y > k | \mathbf{x})} \right) = \theta_k - (\beta_1 Bonding + \beta_2 Linking + \beta_3 SES + \beta_4 Cog + \dots)
\end{equation}
其中,$\beta$ 为回归系数。若 $\beta > 0$,表示随着自变量增加,学生更有可能选择较高的教育期望等级。

\subsection{估计策略}
\begin{itemize}
    \item \textbf{聚类稳健标准误}:考虑到学生嵌套于班级,使用按班级编号 (\texttt{clsids}) 聚类的稳健标准误(Cluster-Robust SE)以校正组内相关性。
    \item \textbf{比例优势假设(PO)检验}:使用似然比检验(LR Test)对比有序Logit与多项Logit(MNLogit)。结果显示 $p \approx 0$,即PO假设被拒绝。鉴于有序模型解释的简洁性,仍将其作为主模型,但以多项Logit结果作为稳健性对照。
\end{itemize}

\section{实证分析结果}

\subsection{主效应分析 ($H_1$)}
主模型回归结果显示,所有核心变量均在 $p < 0.001$ 水平上显著。

\begin{table}[H]
\centering
\caption{教育期望影响因素的主模型回归结果(示意)}
\begin{tabular}{lccc}
\toprule
变量 & 系数 ($\beta$) & 标准误 (Robust SE) & 显著性 \\
\midrule
\textbf{Cog\_Score (Z)} & 0.56 & 0.015 & *** \\
\textbf{SES\_PCA (Z)}   & 0.33 & 0.018 & *** \\
\textbf{Linking\_Idx (Z)} & \textbf{0.22} & 0.016 & *** \\
\textbf{Bonding\_Idx (Z)} & \textbf{0.14} & 0.014 & *** \\
Hukou (农村=1) & 0.11 & 0.045 & * \\
\bottomrule
\end{tabular}
\footnotesize{注:*** p<0.001, ** p<0.01, * p<0.05。控制了性别等其他变量。}
\end{table}

\textbf{结果解读:}
\begin{enumerate}
    \item \textbf{认知主导}:认知能力 ($\beta \approx 0.56$) 是最强的预测因子。
    \item \textbf{社会资本有效}:师生关系 ($\beta \approx 0.22$) 和同伴关系 ($\beta \approx 0.14$) 均显著正向预测教育期望。假设 $H_1$ 得到支持。
    \item \textbf{农村韧性}:在控制其他变量后,农村学生的教育期望略高于城镇学生 ($\beta \approx 0.11$)。
\end{enumerate}

\subsection{交互效应与异质性分析 ($H_2$)}

为了检验社会资本作用的公平性,我们引入了交互项。

\begin{itemize}
    \item \textbf{同伴关系 $\times$ SES}:系数显著为正 ($\beta > 0, p < 0.05$)。
    \begin{quote}
        \textbf{结论:马太效应。} 高SES学生能从良好的同伴关系中获得比低SES学生更大的期望提升。这表明同伴网络可能在某种程度上放大了阶层优势。
    \end{quote}
    
    \item \textbf{师生关系 $\times$ SES}:系数不显著 ($p > 0.1$)。
    \begin{quote}
        \textbf{结论:} 师生关系对不同家庭背景学生的作用差异不明显。
    \end{quote}
    
    \item \textbf{师生关系 $\times$ 户籍 (Rural)}:系数边缘显著为正 ($p \approx 0.05 \sim 0.16$)。
    \begin{quote}
        \textbf{结论:微弱补偿效应。} 相比城镇学生,农村学生似乎能从师生关系中获得略多的边际收益。
    \end{quote}
\end{itemize}

\subsection{非线性检验与机器学习探索}
\begin{itemize}
    \item \textbf{非线性检验}:使用样条函数发现,同伴关系呈线性特征,而师生关系存在一定的非线性信号 ($p \approx 0.005$)。
    \item \textbf{随机森林验证}:特征重要性排序为 $Cog > SES > Linking > Bonding$,与Logit模型系数大小一致。部分依赖图(PDP)显示了自变量与高教育期望概率的单调正向关系。
\end{itemize}

\section{结论与讨论}

\subsection{主要发现}
本研究证实,学校不仅是知识传授的场所,更是社会资本积累的重要场域。
\begin{enumerate}
    \item \textbf{社会资本是独立资源}:即便家庭贫困,良好的师生和同伴关系也能为学生提供向上的动力。
    \item \textbf{同伴关系的“双刃剑”}:同伴网络更倾向于“锦上添花”,高阶层学生更能利用这一资源(马太效应)。
    \item \textbf{教师的关键角色}:相比同伴,师生关系更具普适性,且对农村学生具有潜在的补偿价值。
\end{enumerate}

\subsection{政策建议}
\begin{itemize}
    \item \textbf{强化导师制}:对于农村及低SES学生,单纯的同伴融合可能不足以弥补差距,应优先加强教师对其的一对一支持(Mentoring)。
    \item \textbf{警惕圈层化}:在班级建设中,需注意防止高SES学生的同伴圈层化导致的优势垄断。
\end{itemize}

\subsection{研究局限}
\begin{enumerate}
    \item \textbf{因果推断}:横截面数据无法完全排除反向因果(如高期望学生主动寻求好关系)。
    \item \textbf{测量主观性}:社会资本基于学生自评,缺乏客观社交网络数据。
    \item \textbf{样本限制}:剔除了“无所谓”群体,结论外推需谨慎。
\end{enumerate}

\end{document}
